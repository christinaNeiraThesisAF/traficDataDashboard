\documentclass{kththesis}



\titleformat
{\chapter}
[display]
{\normalfont\Large\bfseries}
{Kapitel \thechapter}
{0.5ex}
{}
[]



\usepackage{blindtext} % This is just to get some nonsense text in this template, can be safely removed
\usepackage{caption}
\usepackage{multirow}

\usepackage{csquotes} % Recommended by biblatex
\usepackage{biblatex}
\addbibresource{references.bib} % The file containing our references, in BibTeX format


\title{This is the English title}
\alttitle{Detta är den svenska översättningen av titeln}
\author{Osquar Studnt}
\email{osquar@kth.se}
\supervisor{Lotta Larsson}
\examiner{Lennart Bladgren}
\programme{Master in Computer Science}
\school{School of Electrical Engineering and Computer Science}
\date{\today}

\begin{document}


% Frontmatter includes the titlepage, abstracts and table-of-contents
\frontmatter

\titlepage

\begin{abstract}
  English abstract goes here.

  \blindtext
\end{abstract}


\begin{otherlanguage}{swedish}
  \begin{abstract}
    Träutensilierna i ett tryckeri äro ingalunda en oviktig faktor,
    för trevnadens, ordningens och ekonomiens upprätthållande, och
    dock är det icke sällan som sorgliga erfarenheter göras på grund
    af det oförstånd med hvilket kaster, formbräden och regaler
    tillverkas och försäljas Kaster som äro dåligt hopkomna och af
    otillräckligt.
  \end{abstract}
\end{otherlanguage}


\tableofcontents


% Mainmatter is where the actual contents of the thesis goes
\mainmatter


\chapter{Inledning}


\section{Problemformulering}

Ökade mängder av strömmande data som uppstått i samhället är ett resultat av övervakning av bland annat trafik, väder, statistik från hemsidor, sociala medier samt servrar och databaser. Att presentera dessa dataströmmar på interaktiva hemsidor gör att människor kan få önskad information för att kunna förstå situationer. Dock har det visat sig vara en utmaning att presentera strömmande data i realtid, i exempelvis en webbapplikation med en mängd olika element med data (eng. Dashboard). Svårigheten ligger i att göra information så lättförståelig för användaren som möjligt och beror på en mängd faktorer. Det handlar om vilka datavisualiseringselement som ska användas och hur de ska anpassas för att underlätta förståelsen. Applikationen som presenterar datat måste därför vara designad på ett sätt så att användaren förstår: 


\begin{itemize}
\item Vad det är som presenteras
\item Vilka handlingar som bör utföras för att 	   utvinna den önskade informationen
\item Vad som kommer hända efter att en handling 		utförts av användaren
\end{itemize}

För att presentera datamängder finns det en mängd vanligen använda element, såsom linjediagram, stapeldiagram, cirkeldiagram samt punkter eller regioner i färg i kartor. Vid tillämpning av dessa element finns det egenskaper som kan påverka förståelsen av datat. Några av dem är vilket intervall data kommer presenteras i, hur snabbt data kommer uppdateras och hur länge data kommer vara synligt innan det försvinner. Även estetiska egenskaper som färg, storlek och placering är viktigt vid presentation av element. 

Detta projekt gick alltså ut på att ta fram ett sätt att presentera strömmande data så att de blev användbara, vilket enligt en definition i boken Human computer interaction skriven av Alan Dix, Janet Finlay, Geregory D. Abowd, Russell Beale[1] är att utföra en handling enkelt och naturligt utan risk för fel. Den användare som visualiseringen riktade sig till var den genomsnittliga invånaren utan hög teknologisk kunskap eller erfarenhet inom tolkning av strömmande data. Projektet utfördes genom en analys av de olika datavisualiseringselement som fanns tillgängliga för att presentera data, med elementens egenskaper i åtanke. Dataströmmen som användes som exempeldata togs från Transportstyrelsens öppna databas gällande betalstationer runt om i Sverige. Presentationen av denna data gjordes i en interaktiv webbapplikation följandes principerna inom Människa-Dator interaktion. Oberoende av vilket data som ska användas/presenteras ska metoden som togs fram i denna rapport kunna tillämpas.


\section{Målsättning}
Målet med det här projekt var att genomföra en litteraturstudie kring visualisering av stora mängder strömmande data så att användare enkelt skulle kunna förstå vilka data presenteras, vad som skulle kunna göras med dem och hur de skulle kunna användas. Det handlade alltså om att skapa ett sätt att presentera strömmande data där användare skulle kunna dra nytta av dem. Ett grundläggande mål för projektet var att skapa ett verktyg för testning och tillämpning av analysen. Detta verktyg skulle vara en webbapplikation där data från Transportstyrelsens betalstationer skulle presenteras. Projektet skulle ytterligare delas upp i följande mer detaljerade delmål:


\begin{itemize}
\item Förstudie på ramverk för strömmande analys och visualisering som passade för stora mängder data:
\begin{itemize}
\item Vilka teorier om människa-datorinteraktion fanns, vilka och hur skulle de implementeras?
\item Vilka system fanns tillgängliga?
\end{itemize}
\item Analysera och visualisera data för att skapa en interaktiv dashboard som skulle möjliggöra användarna att utforska datadynamisk genom att följa en design och utvecklingsprocess från människa-datorinteraktion.
\item Genomföra en analys angående representation av stora mängder strömmande data på ett sådant sätt att användare skulle kunna dra nytta av dem och använda dem på det bästa möjliga sättet genom att:
\begin{itemize}
\item Anpassa och analysera vanligen använda element, såsom olika typer av diagram och grafer, vid presentation av data.
\item Dra slutsatser kring tillämpning av element med hjälp av djupintervjuer och processer inom Människa-dator interaktion.
\end{itemize}
\end{itemize} 

\section{Avgränsningar}
Data som skulle användas för analys och visualisering skulle vara data från Transportstyrelsens betalstationer. Skapandet av webbapplikationen, analys och visualisering av data skulle göras med hjälp av Elastic Stacks “open source” projekt: Logstash, Elasticsearch och Kibana. 


\section{Författarnas bidrag till examensarbetet}
Detta är författarnas bidrag till examensarbetet.


\chapter{Teori och bakgrund}

Detta kapitel presenterar bakgrund och teorier till hur ett användargränssnitt designas med både användarupplevelsen och strömmande data i fokus. 

\section{Visualisering} 
\subsection{} 

\begin{figure}[hbtp]
\centering

\caption{Systemarkitektur}
\end{figure}



\chapter{Metoder}
Metoder och resultat text
\blindtext

\captionsetup[table]{name=Tabel}

\begin{table}[h!]
  \begin{center}
    \caption{En passage och dess egenskaper efter modifiering.}
    \label{tab:table1}
    \begin{tabular}{|p{3cm}|p{3cm}|p{3cm}|p{3cm}|}
         \hline
      \textbf{Fält} & \textbf{Format} & \textbf{Exempel} & \textbf{Beskrivning}\\
      \hline
      Datum & YYYY-MM-DD & “February 27th 2015” & Datum för passage\\ % <--
        \hline
      Klockslag & HHMM& “06:56” & Klockslag för passage\\ % <--
        \hline
      SkatteObjekt & Sträng & “GBG”
 & Område för avgift/skatt\\ % <--
        \hline
      Betalstation & Sträng & “Hjalmar Brantingsgatan” & Betalstation\\ % <--
        \hline
      Riktning & Sträng & “Ut” & Körriktning\\ % <--
        \hline
      Län & Sträng & “Västra Götalands Län” & Län\\ % <--
        \hline
      Kommun & Sträng & “Göteborg” & Kommun\\ % <--
        \hline
      Körfältsnummer & Nummer & “4” &  Körfältet som fordonet var i vid passage\\ % <--
        \hline
          Postnr & Sträng & “418xx”
 & Postnr - endast de 3 första siffrorna\\ % <--
        \hline
          Fordonstyp & Sträng & “PERSONBIL” & Fordonstyp\\ % <--
        \hline
          TidStämpel & yyyy-MM-dd'T'HH:mm:ss
 & “February 27th 2015, 07:56:03” & Tidsstämpel för passagen med datum och tid, på sekundnivå 
\\ % <--
        \hline
Geografisk Placering & Sträng & “57.718564, 11.994575” & 
Koordinat för betalstationens 
\\ % <--
        \hline
    \end{tabular}
  \end{center}
\end{table}

\chapter{Resultat}


 \begin{tabular}{|p{3cm}|p{3cm}|p{3cm}|p{3cm}|}
      \hline
     \textbf{Designkrav} & \textbf{Lösningar} & \textbf{Visualisering} & \textbf{Fält}\\
     \hline
  \multirow{3}{3cm}{Hur många bilar åker igenom varje betalstation och/ eller stad?} &  Trafikintensitet baserad på stad & Taggmoln  & SkatteObjekt\\\cline{2-4}
  & \multirow{2}{3cm}{Trafikintensitet baserad på betalstationer} & Karta med koordinater
& Geografisk Placering\\\cline{3-4}  
  & &  Taggmoln
& Betalstationer \\ \hline


  \multirow{2}{3cm}{Varifrån kommer bilar som åker genom betalstationerna?} &  Kommun där fordonsägare bor & Karta med geografiska områden  & Kommun\\\cline{2-4}
  &Län där fordonsägare bor
 & Karta med geografiska områden  & Län\\ \hline
 
Hur många fordon som passerade den senaste minuten? &Antal fordon
 & Antal  & PassageObjekt\\ \hline
 
 \multirow{2}{3cm}{Hur ser trafikfördelningen ut baserad på riktning?} &  Riktnings-fördelning&Cirkeldiagram& Riktning\\\cline{2-4}
  &Körfälts-fördelning 
 & Cirkeldiagram  & Körfält\\ \hline
 
 \multirow{2}{3cm}{Vilken är den mest populära fordonstypen?} &  Trafikintensitet baserad på fordonstyp & Cirkeldiagram& Fordonstyp\\\cline{2-4}
  &Trafikintensitet baserad på fordonstyp för de mest populära betalstationerna 
 & Stapeldiagram  & TidsStämpel, Betalstation, Fordonstyp\\ \hline


\end{tabular}


\chapter{Analys och diskussion}
Analys och diskussion
\blindtext


\chapter{Slutsatser}
Slutsatser
\blindtext

\chapter{Källförteckning}
Källförteckningen.


\chapter{Bilagor}


\end{document}
